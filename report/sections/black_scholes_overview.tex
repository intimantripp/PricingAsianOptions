The Black-Scholes model provides a foundational framework for pricing derivative securities, 
including options, under specific market assumptions. These assumptions include frictionless markets 
(no transaction costs or taxes), continuous trading, unlimited short selling, absence of arbitrage, constant volatility, a known
 and constant risk-free interest rate, and the underlying asset price following a geometric Brownian motion. 
 Under these conditions, Black and Scholes \cite{black1973pricing} showed that the price $V(S, t)$ of any derivative 
 dependent on the underlying asset price $S$ must satisfy the Black-Scholes partial differential equation (PDE):

\begin{equation}\label{eq:black_scholes_pde}
\frac{\partial V}{\partial t} + \frac{1}{2}\sigma^2 S^2 \frac{\partial^2 V}{\partial S^2} + rS\frac{\partial V}{\partial S} - rV = 0,
\end{equation}

where $\sigma$ is the volatility of the underlying asset, $r$ is the risk-free interest rate, and 
$t$ denotes time. This PDE is central to option pricing theory, as solving it—subject to appropriate boundary and
terminal conditions specific to the derivative—enables the determination of the derivative's price at any time prior to maturity.

An alternative but equivalent framework for derivative pricing is the risk-neutral valuation approach. Under this approach, 
the price of a derivative is calculated as the expected discounted payoff under a risk-neutral probability measure 
$\mathbb{Q}$, which effectively eliminates the necessity of incorporating investor risk preferences explicitly. 
Specifically, the value $V(S,t)$ at time $t$ is given by:

\begin{equation}\label{eq:risk_neutral_valuation}
    V(S,t) = e^{-r(T - t)} \mathbb{E}^{\mathbb{Q}}[P(S_T) \mid S_t = S],
\end{equation}

where $P(S_T)$ represents the payoff of the derivative at maturity $T$, and $\mathbb{E}^{\mathbb{Q}}$ denotes the
expectation under the risk-neutral measure. Under the risk-neutral measure the expected return of the
underlying asset is equal to the risk-free rate $r$.

Importantly, pricing a derivative using this risk-neutral expectation is mathematically equivalent to solving the Black-Scholes 
PDE in equation \eqref{eq:black_scholes_pde}. Both methods yield the same theoretical price under the assumptions of the 
Black-Scholes framework \cite{hull2016options}.

In practice, the risk-neutral valuation approach lends itself naturally to numerical techniques such as Monte Carlo simulation. 
In Monte Carlo pricing, we simulate multiple paths of the underlying asset's price evolution under the risk-neutral measure, 
calculate the derivative payoff for each simulated path at maturity, and then take the discounted average of these payoffs to 
approximate the derivative's price. This method is especially beneficial when dealing with complex derivative 
payoffs that do not admit closed-form solutions or are challenging to handle analytically.

Thus, we have two viable numerical approaches to pricing Asian options under the Black-Scholes assumptions:
\begin{enumerate}
\item Employing Monte Carlo simulation within the risk-neutral valuation framework.
\item Solving the corresponding Black-Scholes PDE numerically via finite-difference methods.
\end{enumerate}

We will introduce and compare specific approaches to pricing arithmetic Asian options that utilise each of these approaches
in the subsequent sections.