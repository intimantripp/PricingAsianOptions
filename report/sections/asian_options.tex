The key characteristic of Asian (or average price) options is that their payoff depends on the average price of the underlying
asset during the lifetime of the option \cite{hull2016options}. The payoffs $P(S, K)$, where $K$ is the strike and $S$ the asset value,
of an average price call and average price put are given respectively by:

\begin{subequations}\label{eq:asian_payoffs}
    \begin{align}
        &\text{Average price call: } P(S, K) = \max(0, \bar{S} - K) \label{eq:asian_call_payoff}\\
        &\text{Average price put: }  P(S, K) = \max(0, K - \bar{S}) \label{eq:asian_put_payoff}
    \end{align}
\end{subequations}

where $\bar{S}$ denotes the average asset price. The appeal for such options is that this averaging mechanism makes them less 
suspectible to price movements in the underlying as the option approaches maturity. 
This makes them less susceptible to price manipulation of the 
underlying asset \cite{kemna1990pricing}, something that was a concern when they were first written by 
Banker's Trust Tokyo office as contracts on crude oil contracts \cite{falloon1999evolution}. Asian options are now popular in currency markets,
where many accounting standards require translation of assets or liabilities priced in foreign currencies at an average rate 
over the reporting period. Consequently, Asian options can hedge against against book valuation changes due to FX movements,
and they continue to be common in volatile commodity markets.

There are variations on the Asian options described by equations \eqref{eq:asian_payoffs}. Average strike options use the average 
asset price over the option's lifetime as the strike price, and the payoff depends on the difference between this average strike and 
the underlying asset's terminal value. We do not consider average strike options here. Additionally, Asian options can differ based
on their exercise times. European Asian options have a single exercise opportunity at maturity, while American Asian options allow
exercise at any point before maturity. In this report, we focus exclusively on European Asian options. 

The averaging method commonly employed is the arithmetic average. In the continuous monitoring case, this is:

\begin{equation}\label{eq:continuous_arithmethic_average}
    \bar{S} = \frac{1}{T-T_0} \int_{T_0}^{T}S(t)dt
\end{equation}

where $T$ is the maturity of the option and $T_0$ its issuance time. Without loss of generality we assume $T_0 = 0$ throughout.
For discrete monitoring, where the asset price is measured $n$ times at intervals $t_i$ defined by $t_i = i \frac{T}{n}$ with $t_0 = 0$, 
the arithmetic average becomes:

\begin{equation}\label{eq:discrete_arithmethic_average}
    \bar{S} = \frac{1}{n} \sum_{i=1}^{n}S(t_i).
\end{equation}

This could correspond to daily measurements of the asset's closing price, for instance.
Finally, the alternative averaging method is the geometric average. In the continuous case, this is given by:

\begin{equation}
    \bar{S} = e^{\frac{1}{T}\int_{0}^{T}\ln(S(t))dt}
\end{equation}
 

Although geometric averaging is less commonly used, it remaisn valuable to consider since a closed form solution for 
Asian options with geometric averaging can be obtained under the Black-Scholes framework \cite{kemna1990pricing}.
This closed form solution will be leveraged in two ways in this report. Firstly, we will employ it in verifying that 
our Monte Carlo valuations are accurate and Secondly, it will be utilised in a variance reduction technique for the Monte Carlo
approach, both detailed in section \ref{sec:MC_pricing}. 
