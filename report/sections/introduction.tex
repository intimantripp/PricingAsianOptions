Asian options are financial derivatives whose payoff depends on the average price of the underlying asset throughout the 
lifetime of the option \cite{hull2016options}.
The averaging can be based on either the geometric or arithmetic mean, although arithmetic
Asian options arethe ones predominantly traded \cite{kemna1990pricing}. No analytic pricing formula 
for arithmetic Asian options exists however, and consequently numerical methods are required
to estimate their prices.
In this report, we investigate and compare two widely used numerical approaches for pricing
arithmetic Asian options: a Monte Carlo method and a finite difference method. The evaluation
will focus on computational cost, convergence behavior, and accuracy of these techniques. 
Specifically, for the Monte Carlo simulations, we incorporate variance reduction strategies—antithetic 
variates and control variates—as described in \cite{hull2016options} and \cite{kemna1990pricing} respectively. 
For the finite difference method, we employ a PDE-based approach, solving a transformed version of 
the Black-Scholes PDE using the finite difference scheme introduced by Večer \cite{vecer2001new}.

The report is structured as follows. Section \ref{sec:preliminaries} introduces Asian options and 
provides the relevant background on option pricing through Monte Carlo simulation and PDE techniques. 
In Section \ref{sec:MC_pricing}, we detail the Monte Carlo and various Variance Reduction methods implemented. Section \ref{sec:FD_pricing}
describes the finite difference scheme adopted. Finally, Section \ref{sec:comparison} presents a comparative analysis of the numerical results
obtained from these two methods.

